\chapter{Conclusions}\label{ch:concl}
While the ideas and techniques used in neural networks has been around for nearly 70 years now, it was the developments of the recent years that made neural networks to what they are. With the `re-invention' of neural networks, and the introduction of convolutional neural networks, their strength has been increased significantly. You cannot open a newspaper nowadays and find yet another article about some AI application that has recently been introduced. 99\% of those applications use (some form of) neural networks. 

In this course we have tried to give a broad foundation to the technology of neural networks and deep learning, but also tried to paint the bigger picture of what AI represents and what can be achieved with it. There are many methods that incorporate some form of AI, and their application is wide-spread. Many companies nowadays agree that they should use some form of AI (most likely, some form of data processing through machine learning), yet only a few companies know exactly where to implement AI and how to use it optimally. There are still many challenges ahead for AI, and the correct application of it is one of its biggest.

The basic principles that we tried to teach in this course should remain true for most of the newer applications and developments within the field of AI. Only recently, Google researchers announced a new form of neural networks, called \textit{Capsule networks} \cite{capsules}, which is meant to change the way image processing is handled by neural networks. While it makes significant changes to the architecture of `traditional' neural networks, the underlying principles (layers, activations, weights, etc.) remain the same, and can easily be comprehended with the materials presented in this course.
